\documentclass[12pt]{article}
\usepackage{graphicx}
\title{MAKERERE UNIVERSITY
COLLEGE OF COMPUTING AND INFORMATION SCIENCES
SCHOOL OF COMPUTING AND INFORMATICS TECHNOLOGY}
\author{GROUP: 4}
\begin{document}
\maketitle
\begin{table}[h!]
\begin{center}
\begin{tabular}{|c|c|c|}
\hline
\textbf{Name} & \textbf{StdNo.} & \textbf{RegNo.}\\
\hline
EBOKU EYAYU COLIN & 216002573 & 16/U/4686/PS\\
\hline
KUSIMAKWE MARTIN & 216016978 & 16/U/6385/EVE\\
\hline
WALUGEMBE MARTIN ALVIN & 216005016 & 16/U/12262/EVE\\
\hline
LUBEGA HENRY & 216008631 & 16/U/6602/EVE \\
\hline
\end{tabular}
\end{center}
\end{table}
\newpage
\begin{center}
•\section*{AUTOMATIC SYSTEM FOR MANAGEMENT AND BOOKING OF ADVERTS A NEWVISION}
\maketitle
\end{center}
%\section{Abstract}
%Transaction and record keeping system plays a vital role in financial transactions of a given business organisation. %When evaluating risks, it is more effective to analyse potentail risks.Therefore it is important for an organisation %to manage its risks, enhance its values and improve business perfomance.More so the kind of system used to process the %transactions also greatly determines the rate at which the business is exposed to risk. 
\section{Introduction}
New vision is Uganda's leading press company. 
It comprises many divisions that include T.V radio, newspaper, printing etc. The focus of this project was the advertising department.
There is a high demand for space in advertising especially in the newspaper. However, there existed a challenge in the speed and ease with which advert slots may be booked and managed hence the need for an automatic system where the user can easily book a slot and monitor progress. This would reduce the turn around time and the stress in the overall process of booking an advert.\\
\subsection{Problem statement}
Manual booking is a very hectic process that can leave one rather confused. The process includes contacting a sales agent who would then take your details and contact a designer to make the design who would then have it proofed before it can be printed. In all this, the client has to keep calling or coming physically to the offices to follow up. This system has caused alot of disappointments as regards the date when the adverts runs sine it is not possible for the sales executive to monitor and know the number of slots available for the days paper. In this case the Sales executive is likely to give incorrect information to the customer and raise expectation
\subsection{Objectives}
\subsubsection{General objectives}
To develop a system that enables the client to book an advert in the newspaper without having to come to the office.
\subsubsection{Specific objectives}
\begin{itemize}
\item To study the existing system and get a clear understanding of the current processes.
\item To come up with the requirements and design for a new online system for advert booking.
\item To implement the design of the proposed system.
\end{itemize}
\section{Literature Review} 
A TPRS is Transaction Processing and Record Keeping System. It is a combination of two system i.e. TPS (Transaction Processing System) and the Record Keeping System.
Homogenously these two systems go hand in hand and implantation of one requires some information from the other.\\\\
\cite{r2}The essence of a transaction program is that it manages data that must be left in a consistent state. If an electronic payment is made, the amount must be either both withdrawn from one account and added to the other, or none at all. In case of a failure preventing transaction completion, the partially executed transaction must be rolled back by the TPS. \\\\
\cite{r3}Managing transaction in real time distributed computing system is not easy, as it has heterogeneously networked computers to solve a single problem. The complexity is increased in real time applications by placing deadlines on the response time of the database system and transactions processing.\\
\cite{r4}Although you can use an IT system to keep records, you only full automate record keeping when your software tools capture the required information a part of normal work. 
\section{Methodology of the study}
Methodology consists of these methods and procedures used during the research study.Information for this report was sourced from various secondary sources, all listed in the Reference list. In this section, the methods and procedures are categorized under survey methods and system analysis and design.
\subsection{Survey Methods}
Research method in which questionnaires or interview guides are used to gather data about the existing system and the thoughts of its current users.
\subsubsection{Questionnaires}
In this method, we organized a set of questions to some of the customers and staff in order to collect information about the challenges they face at time when trying to book an advert. From the feed back it indicated the all of participants faced problems with the existing manual system.
\subsubsection{Observation}
We achieved this by observing how the current process of booking of adverts. We were then able to know how much time each transaction takes and how long customers are delayed.
\subsection{System Design}
Systems design is the process of defining the architecture, modules, interfaces, and data for a system to satisfy specified requirements.
\subsubsection{How the system works}
The proposed system works in such a way that the client opens the website and goes into the portal for booking adverts. The client then gives details of the advert in either text form or audio form and the date they intend for the advert to be run. 
\section{Conclusion}
%\begin{figure}
%\includegraphics[width=\linewidth]{interface1.png}
%\caption{ Shows the pricing interface}
%\label{fig:Pricing}
%\end{figure}
%\begin{figure}
%\includegraphics[width=\linewidth]{interface2.png}
%\caption{ Shows the ordering interface}
%\label{fig:Pricing}
%\end{figure}
%Figure \ref{fig:Pricing} shows the pricing interface
This final part of this study summarizes the research carried out, including the key finding and their implications. Thus, there are lots of recommendations which must be taken into considerations. These recommendation includes; changing the old manual system into a computerized system, organizing training workshops for employees to develop and improve their skills, and giving privilege to employee. Furthermore, other recommendations include the protection of the infrastructure needed for the new system (Hardware, Software, Users, Servers, Network protocol etc.), and ensuring periodic maintenance of the system, keeping backups of the data in case of unforseen circumstances.
\bibliographystyle{IEEEtran}
\bibliography{references}

\end{document}
